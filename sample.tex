\documentclass{article}
\usepackage[a4paper,margin=0.7in]{geometry}

\usepackage{amsmath}
\usepackage{multicases}

\title{Multicases}
\author{Potato Man}

\begin{document}
\maketitle
\thispagestyle{empty}

\section*{Introduction}
The `multicases` package provides an environment for typesetting cases with multiple columns. This is useful for displaying piecewise functions with additional annotations or explanations in separate columns.

\section*{Usage}
The `multicases` environment takes two optional arguments:
\begin{itemize}
    \item The number of columns (default is 2).
    \item The alignment specification for the columns (default is the first column centered and the rest left-aligned). The alignment characters are `l`, `c`, `r`, and `a` (for \verb*|align|-like behaviour).
\end{itemize}

\begin{verbatim}
\begin{multicases}[<number of columns>][<alignment>]
    <column 1> & <column 2> & <column 3> & ... \\
    ...
\end{multicases}
\end{verbatim}

\section*{Examples}

\subsection*{1 - custom alignment}
\begin{equation}
    f(x) = 
        \begin{multicases}[3][cla]  % 3 columns; first center, second left, third align-like
            x^2 + \sum{a_{x}^3}            & \text{if $x<0$}       & 0~&(\text{negative}) \\[1em]
            \displaystyle\frac{1}{x^2}     & \text{if $x>0$}       & 1/x~&(\text{positive}) \\[1em]
            0                              & \text{otherwise}      & 0~&(\text{zero})
        \end{multicases}
\end{equation}

\begin{verbatim}
\begin{equation}
    f(x) = 
    \begin{multicases}[3][lca]  % 3 columns; first center, second left, third align-like
        \begin{multicases}[3][cla]
            x^2 + \sum{a_{x}^3}        & \text{if $x<0$}  & 0~&(\text{negative}) \\[1em]
            \displaystyle\frac{1}{x^2} & \text{if $x>0$}  & 1/x~&(\text{positive}) \\[1em]
            0                          & \text{otherwise} & 0~&(\text{zero})
        \end{multicases}
\end{equation}
\end{verbatim}

\subsection*{2 - default behaviour}
\begin{equation}
    f(n) = 
        \begin{multicases}
            n/2  & \text{if $n$ is even} \\
            3n+1 & \text{if $n$ is odd}
        \end{multicases}
\end{equation}

\begin{verbatim}
\begin{equation}
    f(n) = 
        \begin{multicases}
            n/2  & \text{if $n$ is even} \\
            3n+1 & \text{if $n$ is odd}
        \end{multicases}
\end{equation}
\end{verbatim}

\end{document}